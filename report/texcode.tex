%(2)
\documentclass[11pt, dvipdfmx]{article}
\pagestyle{plain}
\usepackage{url}
\usepackage{amsmath}
\usepackage{graphicx}
\setlength{\oddsidemargin}{0mm}
\setlength{\evensidemargin}{0mm}
\setlength{\topmargin}{0mm}
\setlength{\textheight}{215mm}
\setlength{\textwidth}{160mm}

\begin{document}
\title{数値計算法Bレポート}
\author{九大太郎}
\date{February 4, 2026}
\maketitle

\section{ディラック方程式}
自由粒子のディラック方程式は
\begin{equation}
  i\gamma^{\mu}\partial_{\mu}\psi(x)-m\psi(x) = 0
\end{equation}
と書ける。ここで\(\psi(x)\)は、
\begin{equation}
  \psi(x)=
  \begin{pmatrix}
  \psi_1(x)\\
  \psi_2(x)\\
  \psi_3(x)\\
  \psi_4(x)\\
  \end{pmatrix}
\end{equation}
のスピノルで、\(m\)は粒子の質量である。
\end{document}


